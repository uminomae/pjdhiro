% !TEX program = lualatex
% main_final.tex (v0.6) -- condensed + cleaned refs
\documentclass[11pt,a4paper]{article}

\usepackage{luatexja}
\usepackage[margin=25mm]{geometry}
\usepackage{graphicx}
\usepackage{float}
\makeatletter
\renewcommand{\fnum@figure}{図\thefigure}
\makeatother
\usepackage{amsmath}
\usepackage{enumitem}
\usepackage[hidelinks]{hyperref}
\usepackage[backend=biber,style=numeric,sorting=none]{biblatex}
\addbibresource{refs.bib}

\title{美駆動思考支援:北極星(方位)と揺らぎの器(保持)の二機能フレームワーク\\
\large Aesthetic-Conflict-Driven Thinking Support: A Two-Function Framework of North Star and Vessel}
\author{pjdhiro}
\date{v0.6}

% --- Terms (minimal) ---
\newcommand{\NorthStar}{北極星}
\newcommand{\Vessel}{揺らぎの器}
\newcommand{\Holding}{保持}
\newcommand{\AestheticConflict}{美的葛藤}

\begin{document}
\maketitle

\begin{abstract}
本稿は、不確実な状況下で「思考が防衛(安心・正当化・説明の増殖)のために作動してしまう」現象を抑え、
揺らぎを保持したまま創造的探索へ進むための、二機能思考支援フレームワークを提案する。
核となる機能は、(i)\NorthStar{}:真理/未知の真理へ近づく\textbf{方向づけ(方位)}を返す(説明ではない)、
(ii)\Vessel{}:\AestheticConflict{}(知りたい$\times$怖い)を消さずに同居させ、変換前段として保持する。
臨床理論(Bionのthinking理論・container/contained、Meltzerの美的葛藤)を一次アンカーとして位置づけ、
読者向けの暫定定義を先に提示し、厳密化は補足として分離する。
本稿では、概念・運用手順・主張(C1--C4)を最小限に整形し、検証設計は今後の課題として整理する。
\end{abstract}

\section{はじめに}
不確実性が高い局面では、思考が「探索」ではなく「防衛」のモードへ滑りやすい。
本稿は、この滑り(説明=安心、物語化=正当化)を抑えつつ、\AestheticConflict{}を崩さずに前へ進むための
支援機能を、\NorthStar{}と\Vessel{}の二つに分解して提案する。
着想の契機は物語作品(アニメ『チ。』)だが、本稿は作品論ではなく、思考支援の概念設計として提示する。
図\ref{fig:reactor}は、作品から抽出した最小のプロセス仮説(伝承→葛藤→保持→美的体験)を可視化したものである。

\section{背景(一次アンカー)}
\subsection{Bion: thinking と container/contained}
Bionは、thoughts(思考内容)とthinking(思考作用)を区別し、考える/知ることが情動的経験として作動することを論じた。
また、未消化経験が思考可能な形へ変換される過程(alpha function等)や、container/containedの枠組みは、
「揺らぎが変換前段で保持されない場合に、防衛的排出へ偏る」という見立ての足場になる \cite{Bion1962XXXX,BionContainerXXXX}。

\subsection{Meltzer: \AestheticConflict{}(知りたい$\times$怖い)}
Meltzerは、美的体験を「未知(真実)に触れる際の好奇心と不確かさ(恐れ)の緊張」として扱い、
この葛藤に持ちこたえることが心の進展や創造性と関係しうる、という視点を与える \cite{Meltzer1988,KyotoPaperXXXX}。

\section{フレームワーク}
\subsection{読者向け定義(前段)}
\begin{itemize}[leftmargin=*,itemsep=2pt]
  \item \NorthStar{}:\textbf{「真理/未知の真理」へ近づく方向づけを、恐れと欲望の混合を崩さずに返す機能。}
  \item \Vessel{}:\textbf{揺らぎを「思考可能な形(言葉・図・手順)」へ変換する前段として保持する機能。}さらに、\AestheticConflict{}の両極を消さずに同居させる保持機能。
  \item \Holding{}:揺らぎを排出せず持ちこたえること。持ちこたえは創造へ、回避は防衛へ倒れうる。
\end{itemize}

\subsection{使い方(運用手順:最短版)}
ここは、\NorthStar{}/\Vessel{}の\textbf{使い分け}と\textbf{手順}を、最短の形で示す。

\paragraph{いつ何を使うか(分岐)}
\begin{itemize}[leftmargin=*,itemsep=2pt]
  \item まず\Vessel{}:頭の中が理由説明で埋まる/今すぐ結論を出したい/誰かに送って安心したい、となったら\textbf{保持に入る}。
  \item 次に\NorthStar{}:保持後、「では次に何を観測・試行するか」だけを\textbf{方位+一手}で決める。
\end{itemize}

\paragraph{手順(3ステップ)}
\begin{enumerate}[leftmargin=*,itemsep=2pt]
  \item \Vessel{}(保持):3分(短)/10分(標準)/30分(長)のいずれかを選び、保持中は\textbf{即時解釈・即時行動・即時共有}をしない。
  \item 最小外部化(1単位):保持終了時に、次のテンプレのいずれか\textbf{1つだけ}を書く。
    \begin{itemize}[leftmargin=*,itemsep=1pt]
      \item 1行テンプレ:\texttt{いまの揺らぎ:\underline{\hspace{30mm}}(知りたい:\underline{\hspace{18mm}}/怖い:\underline{\hspace{18mm}})}
      \item 1手順テンプレ:\texttt{次の観測:\underline{\hspace{30mm}}(場所/時間/対象)}
    \end{itemize}
  \item \NorthStar{}(出力):次のテンプレで\textbf{1行}だけ返す(説明は禁止)。
    \begin{itemize}[leftmargin=*,itemsep=1pt]
      \item \texttt{[方位] \underline{\hspace{28mm}} -> [次の一手] \underline{\hspace{40mm}}}
    \end{itemize}
\end{enumerate}

\subsection{出力仕様(禁止事項つき)}
\begin{itemize}[leftmargin=*,itemsep=2pt]
  \item NS-O1 出力制約:返答は「方位(orientation)」+「次の一手(1 step)」のみ。\textbf{理由説明・結論確定・安心の供給}をしない。
  \item NS-O2 二極保持:返答内に「恐れ(avoidance)」と「欲望(approach)」の要素を最低1つずつ残す。
  \item VS-O1 保持窓:保持中は\textbf{意味づけ・原因究明・結論化}を遅延させる。
  \item VS-O2 排出禁止:保持中は\textbf{共有(同意獲得)・断定・衝動的処置}をしない。
  \item VS-O3 最小外部化:外部化は\textbf{1行/1図/1手順のうち1つ}に限定する。
\end{itemize}

\subsection{概念図}
\begin{figure}[H]
  \centering
  \includegraphics[width=\linewidth]{img.jpg}
  \caption{精神分析的「美」の生成炉(概念図):伝承→葛藤→保持を経て、未消化情動が「思考が生まれる場所」としての美的体験へ接続する(プロセス仮説)。}
  \label{fig:reactor}
\end{figure}

\section{主張(C1--C4)}
\begin{itemize}[leftmargin=*,itemsep=2pt]
  \item[C1] thinking(思考作用)/knowing(知ること)は情動的経験として作動する(Bion)。 \cite{Bion1962XXXX,BionContainerXXXX}
  \item[C2] 未消化経験は変換前段(保持・コンテインメント)を要し、保持されない場合は防衛的排出へ偏りうる(Bion)。 \cite{Bion1962XXXX,BionContainerXXXX}
  \item[C3] 不確実さに耐えられないとき、解釈は安心・正当化のために増殖し、防衛として作動しうる(Bionのリンク攻撃等の系譜)。 \cite{BionContainerXXXX}
  \item[C4] \AestheticConflict{}(知りたい$\times$怖い)にもちこたえることは、探索・創造・発達と関係しうる(Meltzer)。 \cite{Meltzer1988,KyotoPaperXXXX}
\end{itemize}

\section{今後の課題と方向性(最小)}
\begin{itemize}[leftmargin=*,itemsep=2pt]
  \item 作用機序の同定:\NorthStar{}と\Vessel{}が「防衛化(説明の増殖)」をどの程度抑制するかを、行動指標(出力長、断定語、衝動的共有の有無等)で最小検証する。
  \item 境界条件:急性ストレス/高リスク意思決定など、\Vessel{}が有害(遅延が危険)になり得る条件を明示する。
  \item 既存概念との接続:Bion/Meltzer以外(holding, mentalizing, epistemic trust等)との整合・差分を、過剰な統合を避けつつ追記する。
  \item 実装評価:チャットボット等の支援系に埋め込む場合、ユーザーが「安心」を要求する圧力に対して、方位+一手へ留める設計(ガードレール)の検証が必要である。
\end{itemize}

\printbibliography

\end{document}