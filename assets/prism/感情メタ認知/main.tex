% !TeX program = lualatex
% 画像ファイル(同一フォルダに配置):
%   fig_emometacog_v1_4.jpg  (または .png にして拡張子を合わせる)
\documentclass[10pt]{article}

\usepackage[margin=1in]{geometry}
\usepackage{luatexja}
\usepackage{luatexja-preset} % 日本語フォントのプリセット(LuaLaTeX想定)
\usepackage{newtxtext,newtxmath} % Times系(任意)
\usepackage{microtype}
\usepackage{amsmath,amssymb}
\usepackage{booktabs}
\usepackage{graphicx}
\usepackage[numbers]{natbib}

\title{感情メタ認知の最小プロセスモデル:\\
ゲート(脅威/安全×防衛)、観照による制御、外部化された他者モデル更新(AI外注)の統合}
\author{(著者名)}
\date{}

\begin{document}
\maketitle

\begin{abstract}
本稿は、感情メタ認知を「頭脳と心と身体の間」に位置する最小の制御モデルとして定式化する。
中核は(1)脅威/安全(愛着)と防衛機制が、注意・言語化・行動方針へのアクセスを開閉する「ゲート」を形成すること、
(2)予測--誤差--焦点化(サリエンス)を、観照(注意制御・内受容)によって調整できること、
(3)感情ラベリング(タグづけ)は情動反応を低減しうる一方で、固定化すると探索を止めうるため「仮説として扱う最小手順」を組み込むこと、
の3点である。さらに(4)間主観性(共有現実)とAI外注を同一の「外部化された他者モデル更新」として扱い、
認知の外部化がもたらす利得(負荷低減・探索拡張)とコスト(記憶・学習・主体感/所有感の毀損)をトレードオフとして予測可能にする。
最後に、自己観察プロトコルおよび検証可能な予測(ゲート推定、ラベル固定化の境界条件、AI外注の副作用)を提示する。
\end{abstract}

\section{はじめに}
感情メタ認知(自分の感情を「観測し、言葉にし、扱い方を選ぶ」能力)は、臨床・発達・認知心理の各領域で部分的に扱われてきた。
一方で、日常運用では「感情を見ない(回避)」「理屈で覆う(認知的回避)」「ラベルに同一化する(固定化)」などが起きやすく、
統合的な最小モデルがあると実用上の設計(自己観察、対人場面、LLM外注)に役立つ。

Figure~\ref{fig:emometacog-v14} は、本稿が扱う最小モデル(v1.4)の構造とプロセスを図示したものである。
本稿は、既存研究の強いアンカー(affect labeling、愛着と情動調整、防衛機制、マインドフルネス、予測処理、共有現実、認知オフローディング)を接続し、
4つの主張(C1--C4)として最小限にまとめる。

\begin{figure}[t]
  \centering
  \includegraphics[width=\textwidth]{fig_emometacog_v1_4.jpg}
  \caption{\textbf{感情メタ認知(v1.4)}:最小プロセスモデル。
  (i) 脅威/安全(愛着)×防衛によるゲート開閉、
  (ii) 予測--誤差--焦点化(サリエンス)を観照(注意/内受容)で調整するループ、
  (iii) 「ラベルは仮説」とする最小手順(保留$\rightarrow$最小検証$\rightarrow$再評価)、
  (iv) 対人相互作用とLLM外注を同一の「外部化された他者モデル更新」として扱う枠組み。}
  \label{fig:emometacog-v14}
\end{figure}

\section{関連研究(最小)}
\subsection{感情ラベリングと情動調整}
感情を言葉にする(affect labeling)操作は、情動反応の低減と関連づけて議論されてきた \citep{lieberman2007,torre2018}。
ただし、内省的な言語化が判断の質を損ねる場合もあり \citep{wilson1991}、ラベルの使い方(固定化の回避)が論点となる。
また、感情経験の分化(emotion differentiation)は、調整と関連することが示されている \citep{barrett2001}。
情動調整全体はプロセスモデルとして整理されている \citep{gross1998}。

\subsection{脅威/安全(愛着)と防衛機制}
成人愛着は不安・回避などの次元を通じて、過活性化/不活性化といった調整戦略に結びつく \citep{mikulincer2003,shaver2007}。
防衛機制は階層性(成熟度)とともに実証的に扱われ \citep{vaillant1986,vaillant1992}、近年は評価尺度(DMRS系)も整備されている \citep{digiuseppe2021}。
安全/脅威が自律神経と行動レパートリーを切り替える枠組みとして、ポリヴェーガル観点が参照される \citep{porges2007}。

\subsection{観照(マインドフルネス)、内受容、予測処理}
マインドフルネスは注意制御と受容的志向性として操作定義され \citep{bishop2004}、神経科学レビューもある \citep{tang2015}。
予測処理(自由エネルギー原理)は知覚・行為・学習を統一的に扱う \citep{friston2010}。
内受容推論は、身体信号の予測と更新として感情経験を捉える \citep{seth2013}。

\subsection{間主観性、外部化、AI外注}
共有現実(shared reality)は「内的状態を他者と共有しているという経験」として整理される \citep{echterhoff2018}。
対人的な分散記憶(transactive memory)は親密関係などで形成され \citep{wegner1991}、インターネット検索はその外部化を促進しうる \citep{sparrow2011}。
認知的オフローディングは、負荷低減と代償(記憶保持など)の両面を持つ \citep{risko2016}。
拡張された心(extended mind)は、道具が認知の一部になりうるという基礎枠組みである \citep{clark1998}。
LLM利用の学習・主体感への影響について、近年の実証報告も出始めている \citep{kosmyna2025}。

\section{モデル:4つの主張(C1--C4)}
\subsection{変数(最小)}
以下の潜在状態を最小構成として仮定する(拡張は可能)。
\begin{itemize}
  \item $T \in [0,1]$: 脅威活性(高いほど脅威)
  \item $S = 1-T$: 主観的安全(高いほど安全)
  \item $D$: 防衛様式/成熟度(例:成熟$\rightarrow$神経症的$\rightarrow$未熟)
  \item $\varepsilon$: 予測誤差(内受容を含む)
  \item $\sigma$: 焦点化(サリエンス、注意の重み)
  \item $G \in [0,1]$: ゲート開放度(探索・言語化・再評価・柔軟な方針へのアクセス)
\end{itemize}

\subsection{C2:脅威/安全×防衛=ゲートの開閉}
愛着次元に対応する過活性化/不活性化の調整戦略 \citep{mikulincer2003} と、
生理状態・行動レパートリーの切替 \citep{porges2007} を、注意・言語化・方針選択のアクセス制御として「ゲート $G$」に写像する。
防衛機制はこのゲート操作(気づきの抑圧・変形・迂回)として表現し、階層や測定枠を参照する \citep{vaillant1986,digiuseppe2021}。

\subsection{C3:予測--誤差--焦点化を観照で操作}
予測処理の枠組み \citep{friston2010} に基づき、感情経験を予測誤差 $\varepsilon$ とその精度/重み($\sigma$)の相互作用として捉える。
内受容推論 \citep{seth2013} は、身体チャネルがこの更新に深く関わることを正当化する。
観照(注意制御+内受容への開放性)は、主に $\sigma$(焦点化)を介して更新や再評価の前段を調整すると位置づける \citep{bishop2004,tang2015}。

\subsection{C1:ラベルは効くが固定すると危ない(仮説として扱う)}
感情ラベリングは情動反応の低減と関連する \citep{lieberman2007,torre2018}。
一方で、言語化・内省が判断や選好を損ねる場合がある \citep{wilson1991}。
そこでラベルを「結論」ではなく「仮説」として扱い、固定化(同一化・物象化)を避ける最小手順を導入する:

\begin{quote}
\textbf{最小ラベル手順(MLP: Minimal Label Procedure)}\\
(1) \textit{保留(Withhold)}: 身元タグ・断定を保留する。\\
(2) \textit{命名(Name)}: 候補ラベルを1--2個、仮説として置く。\\
(3) \textit{最小検証(Check)}: 身体感覚/文脈/代替仮説で軽く照合する。\\
(4) \textit{再評価(Re-evaluate)}: 改名・撤回・不確実性の保持を許す。
\end{quote}

\subsection{C4:間主観性とAI外注=外部化された他者モデル更新}
共有現実は、内的状態の「共通性」を作ることで確信や真実味を増す \citep{echterhoff2018}。
対人関係では分散記憶(transactive memory)が形成され \citep{wegner1991}、検索環境は「どこにあるか」の記憶を強める \citep{sparrow2011}。
認知的オフローディングは負荷低減と記憶/学習の代償を同時に扱う \citep{risko2016}。
LLM外注はこの外部化を増幅する新しい相手であり、主体感/所有感や記憶保持に影響しうる \citep{kosmyna2025}。
したがって、対人相互作用とLLM外注を同一の「外部化された他者モデル更新」として扱い、
利得とコストのトレードオフを設計(いつ外注し、いつ内製するか)できると主張する。

\section{応用:自己観察とLLM運用}
\textbf{自己用(最小)。}
$G$ が十分に開いているときにのみMLPを使う。
$T$ が高いときは精密なラベリングよりも安全回復(呼吸、身体の落ち着き、環境調整)と観測(観照)を優先する。

\textbf{LLM外注。}
チャット履歴や生成文を「外部化された記憶/他者モデル」とみなし、
(1) 自分の言葉で目的を言い直す、
(2) アイデンティティ含意の強いタグはMLPで保留する、
(3) 最終要約は自分の短文で書き直す、
などの手当てを推奨する。

\section{予測と評価(検証可能な形)}
\begin{itemize}
  \item \textbf{P1(境界条件)}: ラベリングは短期の反応性を下げるが、反復的な断定タグは固定化を強め、探索を減らす(特に高脅威で顕著)。
  \item \textbf{P2(ゲート依存)}: 観照/注意操作は $G$ が中程度以上のときに再評価・方針選択を改善しやすい。強い脅威下では安全回復が媒介する。
  \item \textbf{P3(外注トレードオフ)}: AI/オフローディングは負荷下のパフォーマンスを上げる一方、後の内的想起・学習・所有感判断を低下させる可能性がある。
\end{itemize}

\section{限界と倫理}
本モデルは診断ではなく、自己観察・運用設計のための最小スキャフォールドである。
病理化を避け、臨床適用時は既存の評価枠と専門的判断に委ねる。
AI外注は利便性と依存・主体感の変化を伴いうるため、目的と境界(何を外注しないか)を明示する。

\section{結論}
感情メタ認知は、(i) ゲート(脅威/安全×防衛)、(ii) 予測--誤差--焦点化の観照による制御、
(iii) ラベルを仮説として扱う最小手順、(iv) 外部化された他者モデル更新(対人/AI)としての統合によって、
最小限の変数で運用可能な形にまとめられる。今後は、ゲート推定・境界条件・AI外注の長期影響を実証的に詰める。

\bibliographystyle{plainnat}
\bibliography{refs}

\end{document}
