\documentclass[10pt]{article}
\usepackage[margin=1in]{geometry}
\usepackage{times}
\usepackage{microtype}
\usepackage{textcomp} % for \textquotesingle
\usepackage{amsmath,amssymb}
\usepackage{booktabs}
\usepackage{graphicx}
\usepackage[numbers]{natbib}

\title{A Minimal Process Model of Affective Metacognition:\\
Gating (Threat/Safety $\times$ Defense), Contemplative Control, and Externalized Other-Model Updating (AI Outsourcing)}
\author{(anonymous draft)}
\date{}

\begin{document}
\maketitle

\begin{abstract}
We propose a minimal process model of affective metacognition that bridges ``brain--mind--body'' dynamics with practical protocols for self-monitoring and AI-assisted cognition. The model combines (i) a gating mechanism driven by threat/safety appraisal (attachment) and defense processes, (ii) contemplative control that modulates prediction--error--salience via attention and interoception, and (iii) a label-as-hypothesis procedure: affect labels can down-regulate reactivity yet become harmful when reified. Finally, we treat intersubjectivity (shared reality) and AI outsourcing as the same class of externalized other-model updating, yielding testable trade-offs between cognitive relief and costs to memory, learning, and authorship/ownership.
\end{abstract}

\section{Introduction}
Figure~\ref{fig:emometacog-v14} illustrates the proposed minimal process model and its operational protocol.

Affective metacognition---monitoring and steering one\textquotesingle s emotional processes---often feels like a ``between'' space connecting brain, mind, and body. Existing literatures offer partial lenses: affect labeling reduces reactivity \citep{lieberman2007,torre2018}, attachment organizes safety vs.\ threat regulation strategies \citep{mikulincer2003,shaver2007}, mindfulness operationalizes attention and acceptance \citep{bishop2004,tang2015}, and predictive processing provides a unifying control language \citep{friston2010,seth2013}. Meanwhile, cognition is increasingly externalized to devices and LLMs; this resembles transactive memory and cognitive offloading \citep{wegner1991,risko2016,sparrow2011}, and may affect ownership of writing \citep{kosmyna2025}.

We unify these into a minimal model with four claims (C1--C4 in the accompanying state file). The goal is not a comprehensive theory, but a compact scaffold that (a) maps to measurable constructs, (b) yields practical self-protocols, and (c) supports design decisions for AI outsourcing without over-identifying with ``tags.''

\begin{figure}[t]
  \centering
  \includegraphics[width=\textwidth]{fig_emometacog_v1_4.jpg}
  \caption{\textbf{Affective metacognition (v1.4):} a minimal process model integrating
  (i) threat/safety-driven gating (attachment $\times$ defenses),
  (ii) a prediction--error--salience loop modulated by contemplative observation (attention/interoception),
  (iii) a minimal ``label-as-hypothesis'' procedure (withhold $\rightarrow$ check $\rightarrow$ re-evaluate),
  and (iv) externalized other-model updating via social interaction and LLM outsourcing.}
  \label{fig:emometacog-v14}
\end{figure}

\section{Related Work (Very Brief)}
\textbf{Labeling and differentiation.} Affect labeling can attenuate limbic responses \citep{lieberman2007} and is framed as implicit emotion regulation \citep{torre2018}. Yet verbal analysis can degrade preference quality in some contexts \citep{wilson1991}, motivating boundary conditions and careful label use. Emotion differentiation relates to regulation outcomes \citep{barrett2001} and sits within broader process models of regulation \citep{gross1998}.

\textbf{Threat/safety and defenses.} Attachment orientations are linked to hyperactivating vs.\ deactivating regulation strategies \citep{mikulincer2003,shaver2007}. Defense mechanisms admit hierarchical organization and measurement approaches \citep{vaillant1986,vaillant1992,digiuseppe2021}. Polyvagal theory provides a physiological lens on safety/threat states and behavioral repertoires \citep{porges2007}.

\textbf{Predictive processing and contemplative control.} The free-energy principle frames perception/action as prediction error minimization \citep{friston2010}. Interoceptive inference connects bodily signals to felt emotion and selfhood \citep{seth2013}. Mindfulness is operationalized as attention regulation plus an accepting orientation \citep{bishop2004}, and has surveyed neural mechanisms \citep{tang2015}.

\textbf{Externalization, others, and AI.} Shared reality describes motivation and mechanisms for aligning inner states \citep{echterhoff2018}. Transactive memory shows distributed encoding in close relationships \citep{wegner1991}, and the Internet can behave like an external memory system \citep{sparrow2011}. Cognitive offloading formalizes when and why we outsource cognition \citep{risko2016}. The extended mind argues that external resources can be constitutive of cognition \citep{clark1998}.

\section{Model}
\subsection{Core variables and notation}
We define a minimal set of latent state variables (scalars; extensions are possible):
\begin{itemize}
  \item $T \in [0,1]$: threat activation (higher = more threatened).
  \item $S = 1-T$: perceived safety (higher = safer).
  \item $D$: dominant defense mode/level (e.g., mature $\rightarrow$ neurotic $\rightarrow$ immature).
  \item $\varepsilon$: prediction error (including interoceptive prediction error).
  \item $\sigma$: salience (attention weight) assigned to a channel/content.
  \item $G \in [0,1]$: gate openness (access to exploration, labeling, reappraisal, flexible policy).
\end{itemize}

\subsection{Gating: threat/safety $\times$ defense}
We model a gating function:
\begin{equation}
G = f(S, D) \quad \text{with } \frac{\partial G}{\partial S} > 0,
\end{equation}
and defense-dependent modulation (e.g., deactivation can keep $G$ low by narrowing channels; hyperactivation can keep $G$ noisy/high but unstable). Empirically, attachment insecurity correlates with deactivating vs.\ hyperactivating strategies \citep{mikulincer2003}. Physiologically, safety/threat states reconfigure available behaviors \citep{porges2007}. Defenses provide a clinical language and hierarchy that can be measured \citep{vaillant1986,digiuseppe2021}.

\subsection{Contemplative control: prediction--error--salience loop}
We assume salience allocation $\sigma$ shapes how prediction error $\varepsilon$ is amplified into felt urgency and subsequent policy selection. ``Contemplative control'' denotes deliberate regulation of $\sigma$ (attention) and interoceptive sampling, consistent with operational definitions of mindfulness \citep{bishop2004}. Predictive processing provides the control intuition: changing attention/precision alters the influence of prediction error \citep{friston2010,seth2013}.

A minimal loop:
\begin{align}
\text{(Observe)} &\Rightarrow \varepsilon \\
\text{(Set precision / salience)} &\Rightarrow \sigma \\
\text{(Update / choose policy)} &\Rightarrow \pi
\end{align}
where mindfulness-like operations act primarily on $\sigma$ (and secondarily on the generative model that produces $\varepsilon$).

\subsection{Label-as-hypothesis: a minimal safe procedure}
Affect labels can reduce reactivity \citep{lieberman2007,torre2018}, but reification can freeze exploration or distort preferences \citep{wilson1991}. We propose a minimal procedure that treats labels as hypotheses:

\begin{quote}
\textbf{Minimal Label Procedure (MLP)}\\
(1) \textit{Withhold}: pause; do not finalize identity-level tags.\\
(2) \textit{Name}: propose 1--2 candidate labels as hypotheses.\\
(3) \textit{Check}: verify against body/interoception and context; seek alternatives.\\
(4) \textit{Re-evaluate}: rename or drop; keep uncertainty explicit.
\end{quote}

This aims to preserve the benefits of labeling while reducing fixation.

\subsection{Externalized other-model updating: human and AI as one class}
Let $O$ denote an ``other model'' used for calibration of beliefs/feelings (shared reality) \citep{echterhoff2018}. Interaction with another person or an LLM both instantiate an external update channel:
\begin{equation}
\theta_{t+1} = \theta_t + \eta \cdot U(\text{dialogue with } O),
\end{equation}
where $\theta$ are internal models (self, others, world). This channel overlaps with cognitive offloading and transactive memory: distributing storage/processing can reduce load \citep{risko2016,wegner1991}, but can shift what is remembered (``where'' over ``what'') \citep{sparrow2011}. Recent work suggests LLM interaction can influence perceived ownership of writing \citep{kosmyna2025}, making it essential to specify when outsourcing is beneficial vs.\ costly.

\section{Applications}
\textbf{Self-use (minimal).} Use MLP only when $G$ is sufficiently open; when $T$ is high, prioritize safety restoration (downshift) and basic observation rather than fine-grained labeling.

\textbf{Clinical/developmental connection (optional layer).} Map recurring patterns of low $G$ to attachment and defense profiles, and track shifts in $D$ or $T$ across time using validated instruments \citep{digiuseppe2021}.

\textbf{LLM outsourcing.} Treat prompts and chat history as externalized memory/other-model. Design safeguards:
(i) keep ``authorial core'' notes internal,
(ii) periodically restate goals in one\textquotesingle s own words,
(iii) run MLP before committing to identity-laden labels suggested by the model.

\section{Predictions (Testable)}
\begin{itemize}
  \item \textbf{P1 (Boundary condition):} labeling reduces immediate reactivity, but repeated identity-level labels increase fixation and reduce exploration, especially under high threat (low $S$).
  \item \textbf{P2 (Gate dependence):} mindfulness/attention manipulations improve reappraisal mainly when $G$ is moderately open; under strong threat, effects are mediated by safety restoration.
  \item \textbf{P3 (Outsourcing trade-off):} AI/offloading improves task performance under load but reduces later internal recall and may alter authorship/ownership judgments.
\end{itemize}

\section{Limitations}
This is a minimal scaffold. It does not diagnose, and it abstracts complex constructs (attachment, defenses, predictive processing) into low-dimensional variables. Empirical work should specify measurement models and boundary conditions.

\section{Conclusion}
Affective metacognition can be made tractable by a compact control model: gating (threat/safety $\times$ defense), contemplative control (prediction--error--salience), and safe labeling (hypothesis procedure), extended to modern cognition via externalized other-model updating (human/AI).

\bibliographystyle{plainnat}
\bibliography{refs}
\end{document}
