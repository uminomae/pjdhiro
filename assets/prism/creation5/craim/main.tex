% !TEX program = lualatex

% (4) main.tex  --- Prism貼り付け想定 / LuaLaTeX固定 / ltjsarticle固定
\PassOptionsToPackage{disable}{microtype}

% --- time counters (early; some packages may reference \hour/\minute) ---
\newcount\hour
\newcount\minute

\documentclass[11pt]{ltjsarticle}

% --- Japanese + fonts (TeX Live bundled HaranoAji) ---
\usepackage{luatexja}
\usepackage{luatexja-fontspec}
\usepackage{fontspec}
\setmainjfont{HaranoAjiMincho}
\setsansjfont{HaranoAjiGothic}
\setmainfont{Latin Modern Roman}
\setsansfont{Latin Modern Sans}

% --- layout ---
\usepackage[margin=22mm]{geometry}
\usepackage{graphicx}
\usepackage{booktabs}
\usepackage{amsmath,amssymb}
\usepackage{url}
\usepackage{xurl}
\usepackage{enumitem}
\usepackage{tikz}
\usetikzlibrary{arrows.meta,positioning}

% --- time counters (force \hour/\minute to be count registers) ---
\makeatletter
\newcount\prism@hour
\newcount\prism@minute
\let\hour\prism@hour
\let\minute\prism@minute
\makeatother

% --- bibliography (compile even if refs.bib is absent) ---
\usepackage[
  backend=biber,
  style=authoryear,
  giveninits=true,
  maxcitenames=2,
  maxbibnames=99,
  url=true,
  doi=true,
  isbn=true
]{biblatex}
\IfFileExists{refs.bib}{\addbibresource{refs.bib}}{}

% --- metadata ---
\title{創造・創発の五相スキーマ(場—波—縁—渦—束)と宇宙普遍構造仮説}
\author{pjdhiro}
\date{2026-01-29}

\begin{document}
\maketitle

\begin{abstract}
本稿は、創造/創発を「場→波→縁→渦→束」の五相として記述するスキーマを提示する。
このスキーマは、対話やプロジェクトで生じる混乱・対立・停滞を「誰が正しいか」ではなく
「どの相で何が起きているか」という診断問題へ変換する共通言語として機能することを狙う。
さらに、組織論の“場(Ba)”、boundary objects、非平衡自己組織化、普遍性(くりこみ群、SOC)
といった既存語彙を足場に、五相が物質・生命・社会・精神に跨る類似生成構造として現れるという
研究仮説(宇宙普遍)を定式化する。最後に、操作的定義と反証可能な予測の最小セットを提案する。
\end{abstract}

\section{導入}
創造(創発)はしばしば「ひらめき」や「才能」の語りで説明されるが、その語りは
観測・診断・設計(介入)に弱い。本稿は、創造を生成プロセスとして扱うための
\emph{五相スキーマ}(場—波—縁—渦—束)を提案する。
本稿の位置づけは「仮説提示」であり、強い実証ではなく、
(1) 概念の分節、(2) 既存研究への翻訳、(3) 反証可能な問いの提示、を主目的とする。

\section{関連研究(最小)}
「場(shared context)」に基づく知識創造の議論(例:Ba)や
境界をまたぐ協働を支える boundary objects の議論は、
本稿の「場/縁」を学術語彙へ翻訳する足場となる\parencite{NonakaKonno1998Ba,StarGriesemer1989Boundary}。
また、非平衡系で秩序が立ち上がる散逸構造\parencite{Prigogine1977NobelLecture}、
非平衡パターン形成\parencite{CrossHohenberg1993Pattern}、
普遍性(くりこみ群)\parencite{Wilson1971Kadanoff}、
自己組織化臨界\parencite{BakTangWiesenfeld1987SOC}
は、宇宙普遍仮説(C7)の論理的足場となる。

\section{提案:五相スキーマ(定義)}
五相を以下のように暫定定義する(本稿の中核仮説)。
\begin{description}[leftmargin=2.2em,style=nextline]
  \item[場] 未分化の背景。複数の可能性が共存し、まだ差が「形」になっていない状態。
  \item[波] 差(揺れ)が顕在化する相。相反する揺れが同時に立つ。
  \item[縁] 揺れが接続点(境界)を作り、相互作用(結合)が生まれる相。
  \item[渦] 対立を否定せず保持しつつ、回転的に統合された「まとまり」が立つ相。
  \item[束] 複数のまとまりが干渉し、方向性(集合的な近似線)として残る相。
\end{description}

\subsection{最小図式}
\begin{figure}[h]
\centering
\begin{tikzpicture}[
  node distance=10mm,
  box/.style={draw, rounded corners, inner sep=4pt, align=center},
  arrow/.style={-{Stealth[length=2mm]}}
]
\node[box] (ba) {場\\(field)};
\node[box, right=of ba] (ha) {波\\(fluctuation)};
\node[box, right=of ha] (en) {縁\\(coupling)};
\node[box, right=of en] (uzu) {渦\\(coherent\\structure)};
\node[box, right=of uzu] (taba) {束\\(direction)};
\draw[arrow] (ba) -- (ha);
\draw[arrow] (ha) -- (en);
\draw[arrow] (en) -- (uzu);
\draw[arrow] (uzu) -- (taba);
\end{tikzpicture}
\caption{五相スキーマ(概念図)}
\end{figure}

\section{診断:失敗モード(C4)}
本スキーマは「良い/悪い」の道徳判断ではなく、
生成の停滞を相の問題として扱うための診断語彙である。
例として以下を挙げる(詳細は今後の事例分析で拡張する)。
\begin{itemize}[leftmargin=1.6em]
  \item \textbf{波の硬直}:揺れの片側を「悪」として排除し、「良い」だけを固定化すると、
  過剰化・硬直が起き、結果として「良い」の自滅に向かう。
  \item \textbf{縁の平坦化}:衝突(緊張)が弱すぎると関係強度が育たず、
  均一で平坦な状態が続き、意図せぬ侵略/同調として現れる。
\end{itemize}

\section{翻訳:組織・対話への接続(C5)}
「場」は共有コンテクストとしての Ba(physical/virtual/mental space)へ、
「縁」は境界をまたぐ翻訳と協働を支える boundary objects へ翻訳できる
\parencite{NonakaKonno1998Ba,StarGriesemer1989Boundary}。
この翻訳により、五相を対話設計(場づくり、境界設計、統合プロセス)へ接続できる。

\section{外挿:物理への構造対応(C6)}
非平衡が秩序の源となるという見方(散逸構造)\parencite{Prigogine1977NobelLecture}や
非平衡パターン形成\parencite{CrossHohenberg1993Pattern}は、
「波(揺れ)」から「渦(秩序構造)」が立つ一般像を与える。
また、結合が臨界値を超えると同期が立ち上がるモデルは、
縁→渦(結合→まとまり)を直感的に例示する\parencite{Strogatz2000Kuramoto}。

\section{宇宙普遍仮説(C7):普遍性としての五相}
くりこみ群の観点は、ミクロとマクロで詳細が異なっても
スケール変換で不変な構造が現れうることを示す\parencite{Wilson1971Kadanoff}。
さらに、自己組織化が臨界点へ向かう SOC は「自己組織化+普遍性」を結びつける\parencite{BakTangWiesenfeld1987SOC}。
これらを足場に、本稿は次を研究仮説として提示する:

\begin{quote}
\textbf{(H)} 五相スキーマ(場—波—縁—渦—束)は、物質→生命→社会→精神に跨って
類似生成構造として現れる(宇宙普遍)。
\end{quote}

\subsection{最小の予測(反証可能性の入口)}
今後の実証のため、最低限の予測形を2つ置く:
\begin{enumerate}[leftmargin=1.6em]
  \item \textbf{境界・結合の閾値仮説}:縁の相では、結合強度が閾値を下回ると渦(まとまり)が立ちにくい。
  \item \textbf{硬直のコスト仮説}:波で揺れの片側を早期に排除すると、短期的安定と引き換えに
  長期の適応性(束としての方向転換能力)が低下する。
\end{enumerate}

\section{限界と今後}
本稿は仮説提示であり、(i) 操作的定義、(ii) 事例データ、(iii) 代替仮説との比較が未整備である。
今後は、対話ログのラベリング、ネットワーク指標、情報量の推移などにより五相を観測可能にする。

\section{結論}
五相スキーマは、創造/創発を診断可能なプロセスとして扱うための「地図」である。
既存語彙(Ba、boundary objects、自己組織化、普遍性)への翻訳を通じて、
宇宙普遍仮説を研究プログラムとして提示した。

\printbibliography
\end{document}
